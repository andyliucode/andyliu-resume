%%%%%%%%%%%%%%%%%%%%%%%%%%%%%%%%%%%%%%%%%
% Friggeri Resume/CV
% XeLaTeX Template
% Version 1.0 (5/5/13)
%
% License:
% CC BY-NC-SA 3.0 (http://creativecommons.org/licenses/by-nc-sa/3.0/)
%
% Important notes:
% This template needs to be compiled with XeLaTeX and the bibliography, if used,
% needs to be compiled with biber rather than bibtex.
%
%%%%%%%%%%%%%%%%%%%%%%%%%%%%%%%%%%%%%%%%%

\documentclass[print]{friggeri-cv} % Add 'print' as an option into the square bracket to remove colors from this template for printing

\begin{document}

\header{Andy}{Liu}{} % Your name and current job title/field

%----------------------------------------------------------------------------------------
%	SIDEBAR SECTION
%----------------------------------------------------------------------------------------

\begin{aside} % In the aside, each new line forces a line break
\section{contact}
\href{mailto:andyliucode@gmail.com}{andyliucode@gmail.com}
~
(516) 859 0389
\section{education}
BS Mathematical Sciences
Minor Computer Science
~
Carnegie Mellon University
Pittsburgh, PA
~
Class of 2017
\section{programming} 
Python, C/C++
\section{tools}
Pytorch, Huggingface, \\Pandas, Sklearn, DyNet, \\Keras, Tensorflow, LaTeX
\section{coursework}
machine learning, \\parallel algorithms, \\probability theory, \\real analysis, \\financal engineering, \\functional programming
\section{hobbies}
tennis, speedcubing, \\German-style board games
\end{aside}

%----------------------------------------------------------------------------------------
%	WORK EXPERIENCE SECTION
%----------------------------------------------------------------------------------------

\section{experience}

\begin{entrylist}

\entry
{Bloomberg, LP}
{September 2017 - Present}
{\emph{Senior Research Engineer - AI Dialogue Modeling}
\begin{itemize}
\item Research, develop, and productionize neural network solutions for Natural Language Understanding tasks on financial language text, such as:
\begin{itemize}
\item Intent classification for trader/broker instant messaging workflows
\item Named Entity Recognition+Disambiguation for security mentions
\item Slotfilling for offer detection and extraction
\end{itemize}
\item Internally published a white paper on stratified sampling and evaluation on large-scale datasets after developing such a system for annotation of large conversational datasets.
\item Developed custom tokenizers/normalizers for training word2vec and RoBERTa embeddings from scratch over large, unlabeled, internal, financial text datasets, which outperform publicly available pretrained embeddings on downstream tasks.
\item Identified issues with dataset quality, correlated them with model errors, and developed several human-computation systems (such as evaluation of workers against experts, data provenance databases, and few-shot learning-based auto-annotation) to solve these problems, which led to measurable improvements in model performance.
\item Won first place in an internal hackathon with a prototype to improve configuration and visualization of hypertune jobs, which was later productionized by the platforms team.
\end{itemize}}
%------------------------------------------------
\entry
{Akuna Capital}
{June 2016 - August 2016}
{\emph{Quantitative Trading Intern}
\begin{itemize}
\item Formulated mathematical conditions for automated trading under slippage. Used Python to build a framework for backtesting, signal processing, and model tuning. 
\item Developed high-Sharpe volatility index trading strategies using statistical regression and empirical backtesting. Traded live risk for the last two weeks of the internship. 
\end{itemize}}
%------------------------------------------------
\entry
{Old Mission Capital}
{June 2015 - July 2015}
{\emph{Quantitative Trading Intern}
\begin{itemize}
\item Built tools to perform time series analysis on the historical performance of commodity ETF rebalances during roll periods. 
\item Used statistical computing tools in Python to analyze fill quality of trades and provide visualization of results across different exchanges and time intervals.
\end{itemize}}
%------------------------------------------------
\entry
{Department of Computer Science - Carnegie Mellon University}
{September 2014 - May 2017}
{\emph{Teaching Assistant}
\begin{itemize}
\item Supported 15-251,  "Great Theoretical Ideas of Computer Science", an intermediate discrete mathematics course with an emphasis on theoretical computer science.
\item Supported 15-451, "Algorithms Design and Analysis", a proof-based computer science course centered around the design and analysis of algorithms.
\end{itemize}}
%------------------------------------------------

\end{entrylist}

%----------------------------------------------------------------------------------------
%	WORK EXPERIENCE SECTION
%----------------------------------------------------------------------------------------

\section{hackathons}

\begin{entrylist}
%------------------------------------------------
\entry
{Citadel Datathon}
{March 2017}
{\emph{Runner-up}
\begin{itemize}
\item Second place in a data science competition with a field of 42 teams. Teams were given an open prompt to analyze 2015 Uber ride data in New York City.
\item Used linear regression with L1 regularization (LASSO) to predict Uber pickup demand using NYC NTA zone demographics as features. Also trained GBM and Ridge Regression models.
\item Produced an executive summary with visualizations of ride pickup hotspots, interpretation of learned model parameters, and predictions of underserved zones.
\end{itemize}}
%------------------------------------------------
\entry
{Tartan Data Science Cup}
{February 2017}
{\emph{Winner}
\begin{itemize}
\item Won first place in a data science hackathon with a field of ~30 teams. Teams were tasked with predicting 'bad' loans using borrower characteristics (e.g. annual income, credit score, etc.).
\item Trained GBM, Random Forest, and Logistic Regression models on a loans data set using sklearn in Python. Tuned model hyperparameters with cross-validation. 
\item Presented key results and methodology to a panel of Capital One Data Scientists. 
\end{itemize}}
%------------------------------------------------
\end{entrylist}
\end{document}