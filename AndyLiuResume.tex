%%%%%%%%%%%%%%%%%%%%%%%%%%%%%%%%%%%%%%%%%
% Friggeri Resume/CV
% XeLaTeX Template
% Version 1.0 (5/5/13)
%
% License:
% CC BY-NC-SA 3.0 (http://creativecommons.org/licenses/by-nc-sa/3.0/)
%
% Important notes:
% This template needs to be compiled with XeLaTeX and the bibliography, if used,
% needs to be compiled with biber rather than bibtex.
%
%%%%%%%%%%%%%%%%%%%%%%%%%%%%%%%%%%%%%%%%%

\documentclass[print]{friggeri-cv} % Add 'print' as an option into the square bracket to remove colors from this template for printing

\begin{document}

\header{Andy}{Liu}{} % Your name and current job title/field

%----------------------------------------------------------------------------------------
%	SIDEBAR SECTION
%----------------------------------------------------------------------------------------

\begin{aside} % In the aside, each new line forces a line break
\section{contact}
\href{mailto:andyliucode@gmail.com}{andyliucode@gmail.com}
~
\href{https://andyliucode.github.io}{andyliucode.github.io}
~
(516) 859 0389
\section{education}
BS Mathematical Sciences
Minor Computer Science
~
Carnegie Mellon University
Pittsburgh, PA
~
Class of 2017
\section{programming} 
Python, C/C++
\section{tools}
Pandas, Sklearn, DyNet, \\Keras, Tensorflow, LaTeX
\section{coursework}
machine learning, \\parallel algorithms, \\probability theory, \\real analysis, \\financal engineering, \\functional programming
\section{hobbies}
tennis, blogging, \\German-style board games
\end{aside}

%----------------------------------------------------------------------------------------
%	WORK EXPERIENCE SECTION
%----------------------------------------------------------------------------------------

\section{experience}

\begin{entrylist}

\entry
{Bloomberg, LP}
{September 2017 - Present}
{\emph{Software Engineer - Artificial Intelligence Group}
\begin{itemize}
\item Improved core machine learning methodology by designing data science pipelines and developing infrastructure, e.g. computing confidence intervals for precision/recall estimates,  evaluation of word embeddings, implementing shared-mask dropout for RNNs, etc.
\item Worked on deep learning solutions for Natural Language Understanding for financial language text, especially intent classification for various trader/broker workflows.
\item Advocated machine learning to non-ML engineers by giving talks on useful techniques as well as presenting summaries of recent research papers.
\end{itemize}}
%------------------------------------------------
\entry
{Akuna Capital}
{June 2016 - August 2016}
{\emph{Quantitative Trading Intern}
\begin{itemize}
\item Formulated mathematical conditions for automated trading under slippage. Used Python to build a backtesting framework, perform signal processing, and implement models. 
\item Developed volatility index trading strategies through a combination of statistical regression and empirical backtesting. Traded live risk for the last two weeks of the internship. 
\end{itemize}}
%------------------------------------------------
\entry
{Old Mission Capital}
{June 2015 - July 2015}
{\emph{Quantitative Trading Intern}
\begin{itemize}
\item Built tools to perform time series analysis on the historical impact of commodity ETF rebalances during roll periods. 
\item Used statistical computing tools in Python to analyze fill quality of trades and provide visualization of results across different exchanges and time intervals.
\end{itemize}}
%------------------------------------------------
\entry
{Department of Computer Science - Carnegie Mellon University}
{September 2014 - May 2017}
{\emph{Teaching Assistant}
\begin{itemize}
\item Supported "Great Theoretical Ideas of Computer Science" (15-251), an accelerated discrete mathematics course with an emphasis on theoretical computer science.
\item Supported "Algorithms Design and Analysis" (15-451), a proof-based computer science course centered around the design and analysis of algorithms.
\end{itemize}}
%------------------------------------------------

\end{entrylist}

%----------------------------------------------------------------------------------------
%	WORK EXPERIENCE SECTION
%----------------------------------------------------------------------------------------

\section{competition}

\begin{entrylist}
%------------------------------------------------
\entry
{Citadel Datathon}
{March 2017}
{\emph{Runner-up}
\begin{itemize}
\item Got second place in a data science competition. Analyzed 2015 Uber ride data in conjunction with New York City demographics and NTA zone datasets. 
\item Used linear regression with L1 regularization (LASSO) to predict Uber pickup demand using demographics. Other models considered include Gradient Boosting Machines and Ridge Regression. 
\end{itemize}}
%------------------------------------------------
\entry
{Tartan Data Science Cup}
{February 2017}
{\emph{Winner}
\begin{itemize}
\item Won first place in a data science hackathon using machine learning methods to predict 'bad' loans based on borrower characteristics (e.g. annual income, credit score, etc.).
\item Trained Gradient Boosting Classifier, Random Forest, and Logistic Regression models on a loans data set using sklearn in Python. Tuned model parameters with cross-validation. 
\end{itemize}}
%------------------------------------------------
\entry
{University of Chicago Midwest Trading Competition}
{March 2014}
{\emph{Case Winner}
\begin{itemize}
\item Designed and implemented in Java a high-frequency trading algorithm to turn a profit by making markets in a single security on two exchanges. Our algorithms had to tolerate position limits and laggy data streams, and they needed to be risk-neutral by end-of-day.
\end{itemize}}
%------------------------------------------------
\end{entrylist}
\end{document}